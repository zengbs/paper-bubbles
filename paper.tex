\documentclass[twocolumn]{aastex631}
\usepackage{mathtools}
\usepackage{natbib}
\bibliographystyle{abbrvnat}
\setcitestyle{authoryear,open={(},close={)}}
\usepackage{txfonts}
\usepackage{lipsum, babel}
\usepackage[T1]{fontenc}
\usepackage{ae,aecompl}
\usepackage{xcolor,colortbl}
\usepackage{tikz}
\usepackage{graphicx}
\usepackage{url}
\usepackage{floatrow}
\usepackage{subfigure}
\usepackage{float}
\usepackage{amsmath}
\usepackage{amssymb}
\usepackage{cleveref}
\usepackage{physics}
\usepackage{empheq}
\usepackage{booktabs}
\usepackage{array}
\newcolumntype{R}[1]{>{\raggedleft\arraybackslash}p{#1}}
\newcolumntype{L}[1]{>{\raggedright\arraybackslash}p{#1}}
\usepackage{multirow}
\usepackage[flushleft]{threeparttable}
\usepackage{mathrsfs}
\usepackage{soul}
\pdfminorversion=5

\newcommand{\MyDiamond}[1][fill=black]{
\begin{tikzpicture}[x=1.2ex,y=1.2ex,line width=.1ex,line join=round, yshift=0.0ex]\
\draw  [#1]  (0,.5) -- (.5,1) -- (1,.5) -- (.5,0);
\end{tikzpicture}
}

\newcommand{\MyPlus}[1][fill=black]
{
\begin{tikzpicture}[x=1.5ex,y=1.5ex,line width=0.5ex]
\draw[#1] (0,.5) -- (1,.5);
\draw[#1] (.5,0) -- (.5,1);
\end{tikzpicture}
}

\newcommand{\MyCross}[1][fill=black]
{
\begin{tikzpicture}[x=1.ex,y=1.ex,line width=0.5ex]
\draw[#1] (0,0) -- (1,1);
\draw[#1] (0,1) -- (1,0);
\end{tikzpicture}
}

\newcommand{\MyTriangle}[1][fill=black]
{
\begin{tikzpicture}[x=1.ex,y=1.ex,line width=0.2ex]
\draw[#1] (0,1) -- (1,1);
\draw[#1] (1,1) -- (.5,0);
\draw[#1] (.5,0) -- (0,1);
\fill[#1] (0,1) -- (1,1) -- (.5,0) -- cycle;
\end{tikzpicture}
}

\newcommand{\MySolidLine}[1][fill=black]
{
\begin{tikzpicture}[x=2.ex,y=1.ex,line width=0.5ex]
\draw[#1] (0,.5) -- (2.5,.5);
\end{tikzpicture}
}

\newcommand{\MyDashedLine}[1][fill=black]
{
\begin{tikzpicture}[x=2.ex,y=1.ex,line width=0.5ex]
\draw [#1,dashed] (0,.5) -- (2.5,.5);
\end{tikzpicture}
}

\newcommand{\MyDashedDottedLine}[1][fill=black]
{
\begin{tikzpicture}[x=2.ex,y=1.ex,line width=0.5ex]
\draw [#1,dash dot] (0,.5) -- (2.5,.5);
\end{tikzpicture}
}

\newcommand{\MyDottedLine}[1][fill=black]
{
\begin{tikzpicture}[x=2.ex,y=1.ex,line width=0.5ex]
\draw [#1,dotted] (0,.5) -- (2.5,.5);
\end{tikzpicture}
}

\definecolor{Gray}{gray}{0.85}

\begin{document}
\title{The symmetry problem of the \textit{Fermi} and eROSITA bubbles: A proof-of-concept study}


\author[0000-0002-1868-0660]{Po-Hsun Tseng}
\affiliation{Institute of Astrophysics, National Taiwan University, Taipei 10617, Taiwan}

\author[0000-0003-3269-4660]{H.-Y. Karen Yang}
\affiliation{Institute of Astronomy, National Tsing Hua University, Hsinchu 30013, Taiwan}
\affiliation{Center for Informatics and Computation in Astronomy, National Tsing Hua University, Hsinchu 30013, Taiwan}
\affiliation{Physics Division, National Center for Theoretical Sciences, Taipei 10617, Taiwan}

\author[0000-0002-1249-279X]{Hsi-Yu Schive}
\affiliation{Institute of Astrophysics, National Taiwan University, Taipei 10617, Taiwan}
\affiliation{Department of Physics, National Taiwan University, Taipei 10617, Taiwan}
\affiliation{Center for Theoretical Physics, National Taiwan University, Taipei 10617, Taiwan}
\affiliation{Physics Division, National Center for Theoretical Sciences, Taipei 10617, Taiwan}

\author{Chun-Yen Chen}
\affiliation{Institute of physics, National Taiwan University, Taipei 10617, Taiwan}

\author[0000-0003-2654-8763]{Tzihong Chiueh}
\affiliation{Institute of Astrophysics, National Taiwan University, Taipei 10617, Taiwan}
\affiliation{Department of Physics, National Taiwan University, Taipei 10617, Taiwan}
\affiliation{Center for Theoretical Physics, National Taiwan University, Taipei 10617, Taiwan}


\correspondingauthor{Po-Hsun Tseng}
\email{zengbs@gmail.com}


\keywords{\textit{Fermi} bubbles, eROSITA bubbles, cosmic rays}

\begin{abstract}
% out of the Galactic center on either side of the Galactic disk
\end{abstract}

\section{Introduction}
The detection of the \textit{Fermi} bubbles \citep{Su2012,Ackermann2014,Narayanan2017},\
two large bubbles symmetrically extending about 50 degrees above and below the Galactic plane,
is one of the great discoveries made with \textit{Fermi} Large Area Telescope \citep{Atwood2009}.

The gamma-ray emission of the \textit{Fermi} bubbles is observed in the energy range\
of $1\lesssim E_{\gamma} \lesssim 100$ GeV and has an almost spatially uniform hard\
spectrum, sharp edges and an approximately flat brightness distribution.\

Over the course of a decade,\
the first all-sky X-ray survey with high-spatial resolution \citep{Predehl2021}\
from eROSITA \citep{Predehl2020} also reveals\
a giant hourglass-shaped structure (eROSITA bubbles hereafter) at the GC,\
extending about twice as large as the \textit{Fermi} bubbles.\
The large-scale X-ray structure\
shows that the intrinsic size of the bubbles is 14 kiloparsecs across \citep{Predehl2021},
and displays morphological symmetry about the Galactic plane as the \textit{Fermi} bubbles.

Their symmetry about the GC suggests that they originate\
from powerful energy injections from the GC, possibly related to nuclear star formation\
\citep{PhysRevLett.106.101102,Carretti2013}
or past active galactic nucleus (AGN) activity \citep{Guo2012,Yang2017}.

Early attempts \citep{Sarkar2015,Yang2017,Zhang2020}\
to model the symmetric \textit{Fermi} and eROSITA bubbles (collectively we call the Galactic bubbles)\
assumed that the jet is vertical to the Galactic plane. However,\
observation \citep{Gallimore2006} has proposed that\
there is a lack of preferred orientation of jets with respect to\
the rotation axis of a galatic plane (disc normal, hereafter).\
In contrast, a number of galaxies in which the jets are oblique to the disc normal\
(e.g. NGC 3079, \citealt{Cecil2001}; NGC 1052, \citealt{Dopita2015}),\
including galaxies in which the jets lie in the plane of the disc (e.g. IC 5063, \citealt{Morganti2015}).

To this end, this work introduces a dense thin disc of interstellar medium\
around a jet source to deflects and decelerates the inclined jets,\
in an attempt to resolve the symmetry problem of the Galactic bubbles.\

The main purpose of this study is to use three-dimensional\
special relativistic hydrodynamics simulations\
involving CR jet injections from the central SMBH in the Galaxy to investigate\
whether the \textit{inclined} jet scenario is able to produce\
the \textit{symmetric} Galactic bubbles that are\
consistent with the observed features, including the shape, the\
surface brightness, and the spectra of \textit{Fermi} bubbles \citep{Ackermann2014}\
and microwave haze \citep{Dobler_2008,PlanckCollaborationIX2013}.

This paper is organized as follows.\
In Section \ref{Methodology}, we describe the numerical techniques and initial conditions employed.\
We directly compare the morphology of the Galactic bubbles with observation,\
and also show the simulated gamma-ray and microwave spectra compared with observations\
in Section \ref{Results}.
Finally, we summarize our findings in Section \ref{Conclusions}.

\section{Methodology}
\label{Methodology}
  We used the GPU-accelerated special relativistic hydrodynamics AMR code (\textsc{gamer-sr}) developed at the\
  National Taiwan University\
  (Schive et al. \citeyear{gamer-1}, \citeyear{gamer-2}; \citeauthor{tseng2021} \citeyear{tseng2021})\
  to carry out the simulations of the Galactic bubbles by CR and relativistic fluid injections from the GC.

  The CRs are advected with the thermal gas, and in return the velocities of gas\
  can react to the gradients of the CR pressure via the source term\
  containing spatial divergence of fluid velocities.\

  Although the high-energy CRe ($10$--$100$ GeV) plays a crucial role in reproducing the $\gamma$-ray map\
  within the range of $1$--$100$ GeV, we assume the pressure of CRe is much less than that of gas\
  throughout the simulation so that we , and the \textit{Fermi} bubbles can be outlined against the eROSITA bubbles.

  As stressed by \citet{Yang2012}, CR diffusion has an insignificant effect\
  on the overall morphology of the \textit{Fermi} bubbles,\
  but only sharpens the edges of the simulated bubbles by the interplay between anisotropic CR diffusion\
  and magnetic fields with suppressed perpendicular diffusion across the bubble surface. Moreover,\
  the bubbles should be weak due to adiabatic expansion, and thus the magnetic fields has\
  a little effect on the overall dynamics. For these two reasons,\
  we have ignored the CR diffusion and the magnetic field throughout\
  the simulation.

  We do not simulate the spectral evolution of the CR, and we neglected the cooling and heating processes of CRs,\
  such as energy losses due to synchrotron and inverse Compton emission, and reacceleration in shocks/turbulences.\

  In this approach, we treat CRs as a single species without distinction between electrons and protons,\
  that cannot react to the gas via the application of CRe pressure,\
  and solve directly for the evolution of CR energy density $e_{\text{cr}}$\
  as a function of $\mathbf{r}$ and $t$.\

  Since the relativistic fluid ejected by the jet source\
  is quickly stalled off and slowed down by a dense ISM disc in a short time,\
  and the relativistic fluid\
  accounts for a little minority of total mass inside the simulation box,\
  we still use the Newtonian gravity to attack this problem.

  The governing equations solving the special relativistic ideal fluid\
  including CR advection, and dynamical coupling between the thermal gas and CRs without CR diffusion\
  can be written a succinct form as


  \begin{subequations}
    \label{governing-eq}
    \begin{align}
     &\partial_{t} D+\partial_{j} \left(DU^{j}/\gamma\right)=0,\label{D evolution}\\
     &\partial_{t} M^{i}+\partial_{j} \left(M^{i}U^{j}/\gamma+p_{\text{gas}}\delta^{ij}\right)=\
     -\rho\partial_{i}\Phi,\label{M evolution}\\
     &\partial_{t} \tilde{E}+\partial_j \left[\left(\tilde{E}+p_{\text{gas}}\right)U^{j}/\gamma\right]=0, \label{E evoltion}\\
     &\partial_{t} \left(\gamma e_{\text{cr}}\right) + \partial_{j} \left(e_{\text{cr}}U^{j}\right)=\
     -p_{\text{cr}} \partial_{j} U^{j},\label{D evolution}
    \end{align}
  \end{subequations}


  where the five conserved quantities of gas $D$, $M^{i}$, and $\tilde{E}$ are the mass density,\
  the momentum densities, and the reduced energy density, respectively.\
  The reduced energy density is defined by subtracting the rest mass energy density of gas\
  from the total energy density of gas.\
  $\gamma$ and $U^{j}$ are the temporal and spatial component of four-velocity of gas.\
  $\rho$ is the gas density in the local rest frame defined by $D/\gamma$.\
  $p_{\text{gas}}$ is the gas pressure.\
  $p_{\text{cr}}$ and $e_{\text{cr}}$ are the CR pressure and CR energy density measured in the local rest frame.\
  $\Phi$ is the gravitation potential.\
  $c$ is the speed of light, and $\delta^{ij}$ is the Kronecker delta notation.\
  Throughout this paper, Latin indices run from 1 to 3, except when stated otherwise.\

  The set of \Cref{governing-eq} is closed by using the Taub-Mathews equation of state \citep{Taub,TM_EOS}\
  that approximates the exact EoS \citep{Synge} for ultra-relativistically\
  hot gases coexisting with non-relativistically cold gases.

  \textsc{gamer-sr} adopts a new algorithm \citep{tseng2021} to convert between\
  primitive ($\rho$, $U^{j}$, $p$) and conserved variables ($D$, $M^{j}$, $\tilde{E}$),\
  significantly reducing numerical error caused by catastrophic cancellations\
  that commonly occur within the regions with high Mach number flows. e.g., jet-ISM interaction zones.

  \textsc{gamer-sr} also adaptively and locally reduce the min-mod coefficient\
  \citep{tseng2021} within the failed patch group when performing full-step updates,\
  new patches allocations, and ghost-zone interpolations.\
  In this way, we provide an elegant way to avoid the use of pressure/density floor,\
  being unnatural but widely used in almost publicly available codes.\

  \subsection{The Galactic and Disk Models}
  As a proof-of-concept study, we approximate conventionally axisymmetric stellar potential of Milky Way\
  by a plane-parallel potential that is symmetric about the mid-plane $z=0$\
  in a simulation box size of\
  $14\times14\times28$ kpc, slightly larger than the size of eROISTA bubbles.

  The plane-parallel potential is fixed throughout our simulations and given by
  \begin{equation}
    \Phi_{\text{total}}(z) = \Phi_{\text{bulge}}(z) + \Phi_{\text{halo}}(z),
  \end{equation}
  where
  \begin{equation}
    \Phi_{\text{bulge}}(z)=\
    2\sigma^2_{\text{bulge}}\
    \ln\cosh\left(z\sqrt{\frac{2\pi G\rho_{\text{bulge}}^{\text{peak}}}{\sigma^2_{\text{bulge}}}}\right)
  \end{equation}
  is the potential of an isothermal slab mainly contributed by stars around the Galactic bulge, and\
  $\Phi_{\text{halo}}(z)=v^2_{\text{halo}}\ln\left(z^2+d^2_{\text{h}}\right)$\
  is a plane-parallel dark logarithmic halo potential.

  With the recourse to the isothermal and hydrostatic equilibrium conditions,\
  and assuming the interfaces between isothermal disc and atmosphere are in thermal pressure equilibrium,\
  we can write the steady-state gaseous density distributions,\
  confined in the total potential, of the disc and the Galactic atmosphere as\
  \begin{subequations}
  \begin{align}
     \displaystyle \rho_{\text{isoDisk}}(z) = \rho_{\text{isoDisk}}^{\text{peak}}
     \exp\left[-\frac{\Phi_{\text{total}}(z)}{k_{B}T_{\text{isoDisk}}/m_{\text{p}}}\right]&\label{isothermal-disc-density}\\
     \text{, if $|z| < z_{0}$}& \nonumber \\
     \nonumber\\
     \displaystyle \rho_{\text{atmp}}(z) = \rho_{\text{atmp}}^{\text{peak}}
     \exp\left[-\frac{\Phi_{\text{total}}(z)}{k_{B}T_{\text{atmp}}/m_{\text{p}}}\right]&\label{isothermal-atmp-density}\\
     \text{, otherwise,}& \nonumber
  \end{align}
  \label{disc-atm-sys}
  \end{subequations}
  where $m_{\text{p}}$ is the proton mass,\
  $T_{\text{isoDisk}}$ and $T_{\text{atmp}}$ is the temperature of the isothermal disc and atmosphere,\
  $\rho_{\text{isoDisk}}^{\text{peak}}$ and $\rho_{\text{atmp}}^{\text{peak}}$ is the peak mass density\
  of the disc and atmosphere on the mid-plane $z=0$.

  We tabulate parameters in the first four\
  categories of \Cref{table-parameters},\
  except for $\rho_{\text{atmp}}^{\text{peak}}$ that\
  can be derived from the other known parameters and thermal pressure equilibrium condition\
  on the interfaces $(z=\pm z_{0})$ between the disc and atmosphere.\

  The density profile of \Cref{disc-atm-sys} is shown in \Cref{fig__initial-density-profile}.\
  Beyond the core radius ($\sim 2 \text{ kpc}$) the gaseous density decreases rapidly as a power-law.

%  The density profile of \Cref{disc-atm-sys} is shown in \Cref{fig__initial-density-profile}\
%  and compared to the observed result \citep{Miller_2013} beyond 1 kpc.\
%
%  Note that there is an difficulty in disentangling the contribution\
%  of the Local Bubble, a supernova remnant in which the Solar System is embedded \citep{Snowden1990},\
%  and the contribution from solar wind charge-exchange processes, which produce soft\
%  X-ray emission throughout the Solar System. As a result, the density profile of the Galactic halo remains unclear\
%  \citep{BlandHawthorn2016}.

  \begin{figure}
    \includegraphics[width=\columnwidth]{figures/fig__initial-density-profile.png}
    \caption{The density profile of the isothermal disc (red pluses) and\
             atmosphere (blue crosses) along the positive z-axis.\
             The density distribution is derived from hydrostatic equilibrium,\
             the interface ($z=0.1$) between the isothermal disc\
             and the atmosphere is pressure balanced.}
    \label{fig__initial-density-profile}
  \end{figure}



  To quantify the synchrotron radiation as a function of position, it\
  is essential to start the simulation with a more realistic magnetic\
  field distribution. To this end, for our fixed magnetic field we adopt\
  the default exponential model in \textsc{galprop} \citep{Strong2007}
  which has the following spatial dependence:\

  \begin{equation}
     |\mathbf{B}(R, z)|=B_{0}\exp\left[-\frac{z}{z_{0}}\right]\exp\left[-\frac{R}{R_{0}}\right],
     \label{magnetic-field}
  \end{equation}


  where $R=\sqrt{x^{2}+y^{2}}$, $B_{0}$ is the average field strength at the GC\
  and $z_{0}$ and $R_{0}$ are the characteristic scales in the vertical and radial\
  directions, respectively. We adopt $z_{0} = 2$ kpc and $R_{0} = 10$ kpc, which\
  are best-fitting values in the \textsc{galprop} model to reproduce the\
  observed large-scale 408 MHz synchrotron radiation in the Galaxy.\
  We choose $B_{0} = 50$ $\mu$G based on the observed field strength at the\
  GC \citep{Crocker2010}.

   % The dispersion:
   %   2010-Comparing the statistics of interstellar turbulence in simulations and observations
   %   2011-RELATIVISTIC JET FEEDBACK IN EVOLVING GALAXIES


  \subsection{The Clumpy Multiphase Interstellar Medium}

  A crucial component in our work is the clumpy ISM disc initialized by\
  the publicly available pyFC code
  \footnote{\url{https://pypi.python.org/pypi/pyFC}}.\

  pyFC randomly generates dimensionless 3D scalar field $f(\bold{x})$\ % that is clumpy and porous.
  that obeys the log-normal probability distribution\
  with mean $\mu$ and dispersion $\sigma$,\
  and follows the power-law Kolmogorov spectrum
  \begin{equation}
    D(\bold{k})=\int k^{2} \hat{f}(\bold{k})\hat{f}^{*}(\bold{k})d\Omega \propto k^{-\beta},
    \label{Kolmogorov-spectrum}
  \end{equation}
  where $\hat{f}(\bold{k})$ is the Fourier transform of $f(\bold{x})$.\
  The spectrum $D(\bold{k})$ in the Fourier space is characterized by the power-law index $\beta=5/3$,\
  the Nyquist limit $k_{\text{max}}$, and the lower cutoff wave number $k_{\text{min}}$.
  $k_{\text{max}}$ is one-half of the spatial resolution within disc,\
  and $k_{\text{min}}$ is 375.0, corresponding to the maximum size of an individual clump $\sim 20$ pc.\
  \citet{LA2002} and \citet{Wagner2012} have outlined a detailed procedure\
  for constructing a clumpy scalar field, and we do not repeat here.

  The density of clumpy disc can thus be obtained by taking the scalar products of\
  $f(\bold{x})$ with $\rho_{\text{isoDisk}}(z)$ over all cells within the disc, i.e.,\
  $\displaystyle\rho_{\text{ismDisk}}(\bold{x}) =\
  f(\bold{x}) \rho_{\text{isoDisk}}(z)$.\
  Also, the thermal pressure equilibrium within the clumpy disc implies that the temperature of disc is
  $\displaystyle T_{\text{ismDisk}}(\bold{x}) =\
  T_{\text{isoDisk}}(z)\rho_{\text{isoDisk}}(z)/\rho_{\text{ismDisk}}(\bold{x})$.


  \Cref{table-parameters} summarize the parameters of the clumpy disc and their references.

  On the basis of this setup, we cover the AMR base level with\
  $16\times16\times32$ root cells, refined progressively on the mid-plane $z=0$\
  based on the gradient of mass density.\
  We also restrict the refinement level at 7 within the cold disc so that\
  a molecular cloud can be adequately resolved by approximately 30 cells along their diameter, 20 pc.

  \Cref{fig__numberDensityHistogram} plots the volume filling factor as a function of\
  initial number density within the disc without jet source.\

  \Cref{fig__zoom-in-disc} shows a close-up view of the\
  pressure, temperature, and number density slices
  in the y-z plane through the center of the disc.

  \begin{figure}
      \includegraphics[width=\columnwidth]{figures/fig__numberDensityHistogram.png}
    \caption{The volume filling factor as a function of\
             initial number density within the disc without jet source.\
             The vertical bands from left to right depict the allowable number densities \citep{peak-ism-density} for\
             hot ionized, warm neutral (WNM), warm ionized (WIM), cold nuetral mediums (CNM), and molecular clouds.}
      \label{fig__numberDensityHistogram}
  \end{figure}

  \begin{figure}
    \includegraphics[width=\columnwidth]{figures/fig__zoom-in-disc.png}
    \caption{Close-up view of the initial\
             pressure (top), temperature (middle), and number density (bottom) slices\
             in the y-z plane through the center of the disc.
             }
    \label{fig__zoom-in-disc}
  \end{figure}

\begin{table*}[t]
\raggedright
\caption{Parameters of the disc, atmosphere, and gravitational potential in the simulation.}
\label{table-parameters}
\begin{tabular}{@{}llrc@{}}
\toprule[1pt]\midrule[0.3pt]
Parameter                             & Description                               & Value                                &  Reference                     \\ \midrule
{\bf Static stellar potential }       &                                           &                                      &                                \\
$\sigma_{\text{bulge}}$               & Velocity dispersion of bulge              & 100 km$\cdot$s$^{-1}$                & \citep{velocity-dispersion-MW} \\
$\rho_{\text{bulge}}^{\text{peak}}$   & Peak average density of bulge             & $4\times 10^{-24}$ g$\cdot$cm$^{-3}$ &   N/A                          \\ \hline
{\bf Static dark halo potential }     &                                           &                                      &                                \\
$v_{\text{halo}}$                     &                                           & 131.5 km$\cdot$s$^{-1}$              & \citep{Johnston1995}           \\
$d_{\text{h}}$                        & Core radius                               & 12 kpc                               & \multicolumn{1}{c}{''}         \\ \hline
{\bf Atmosphere }                     &                                           &                                      &                                \\
$T_{\text{\text{atmp}}}$              & Temperature of atmosphere                 & $10^{6}$ K                           & \citep{temperature-MW}         \\ \hline
{\bf Isothermal disc }                &                                           &                                      &                                \\
$z_{0}$                               & Scale height of disc                      & 100 pc                               & \citep{peak-ism-density}       \\
$T_{\text{\text{isoDisk}}}$           & Temperature of disc                       & $10^{3}$ K                           & \multicolumn{1}{c}{''}         \\
$\rho_{\text{isoDisk}}^{\text{peak}}$ & Peak mass density of disc                 & $10^{-23}$ g$\cdot$cm$^{-3}$         & \multicolumn{1}{c}{''}         \\ \hline
{\bf Clumpy disc }                    &                                           &                                      &                                \\
$\dagger$  $k_{\text{min}}$           & Cutoff wave number                        & 375.0                                & \citep{peak-ism-density}       \\
$\mu$                                 & Mean of scalar field                      & 1.0                                  &   N/A                          \\
$\ddag$  $\sigma$                     & Dispersion of scalar field                & 5.0                                  & \citep{Federrath2010}          \\
$\beta$                               & Power law index                           & -5/3                                 &   N/A                          \\ \midrule
\end{tabular}
\begin{tablenotes}
      \raggedright
      \item  $\dagger$  $k_{\text{min}}=375.0$ leads to the size of an individual molecular cloud $\sim 100$ pc.
      \item  $\ddag$ In numerical simulations of turbulence,\
             \citet{Federrath2010} find $\sigma\sim 3.6$ and 35 for solenoidal (divergence-free)\
             and compressive (curl-free) driving force,\
             respectively, so that our adopted value of 5 is closer to their solenoidal result.
    \end{tablenotes}
\end{table*}


% In addition to the gravitational interaction, we ignore other interactions between stars and gases.
% We also ignore the self-gravity of the ISM disc and of the atmosphere.
% We ignore the centrifugal force of Milky Way rotation acting on the bubbles.
%

\subsection{Inclined jet injection}

  Motivated by these observations, we simulate the jet with an inclination angle\
  $45^{\circ}$ with respect to the Galactic plane\
  in order to release the caveat that\
  the jet direction must be perpendicular to the Galactic plane, and in particular to\
  investigate the effect of the jet-disc misalignment on the symmetry of bubbles.


  A few additional quantities are used to characterize the jets:
  the density contrast between the thermal gas contained in the jet source and the ambient gas,\
  $\rho_{\text{jet}}/\rho_{\text{amb}}=10^{-3}$,\
  the temperature contrast, $T_{\text{jet}}/T_{\text{amb}}=2\times10^{4}$,\
  the pressure ratio of CR to gas is 0.18,\
  and the flow 4-velocity inside the jet source $\beta\gamma = 0.6$\
  along the symmetric axis of cylinder.\
  The jet power is thus $3.2\times 10^{42}$ erg/s, resulting in the Eddington ratio 0.008.

  Note that as we inject the jets at the center of the clumpy disc,\
  we define the ambient gas density by the peak density of\
  the isothermal disc on the mid-plane $z=0$ (i.e. $\rho^{\text{peak}}_{\text{isoDisk}}$),\
  as opposed to the \textit{clumpy} density around the jet source, to avoid ambiguous definition.


  The bipolar jets are constantly ejected from a cylindrical source at the beginning of simulation ($t=0$)\
  and suddenly quenched at $t=1.2$ Myr before fully breaking out the disc.\
  Without quenching, the Galactic bubbles at the present time will be asymmetric about the Galactic plane.
  The jet duration (1.2 Myr) allows the total ejected energy to be $1.2\times10^{56}$ erg, bracketed by\
  $8\times10^{55}$ and $1.3\times10^{56}$ erg, estimated by \citet{Predehl2020}.\



  The diameter and height of cylindrical source are 4 pc, leading to the source volume $\sim 50 \text{ pc}^{3}$ is\
  much smaller than that of an individual clump by a factor of about 83.\
  By intentionally reducing the volume ratio of the jet source to an individual clump,\
  we can mitigate the effect of the randomness of the clumps on the bubbles.\
  Moreover, we resolve the jet source with the highest refinement level 11,\
  bringing the finest spatial resolution up to 0.4 pc.\



\section{Results}
\label{Results}

\subsection{Morphology and properties of Galactic bubbles}

 \Cref{fig__jetI5+ismSeed3-45deg} shows\
 the slices of pressure (top), temperature (middile), number density (bottom)\
 at the end of simulation $t=12.39$ Myr.\
 The slices pass through the bipolar jet source injecting along $z=-y$ direction.

 The fiducial run (leftmost column) with the initial condition specified in Section \ref{Methodology}\
 shows that the edge of the outermost bubbles is indeed a forward shock,\
 expanding 12.5 kpc above and below the Galactic plane,\
 with a semiminor axis about 6.8 kpc on the plane.\
 We note that the overall extent of the outermost bubbles is comparable to\
 the two spherical objects of a radius 6-7 kpc proposed by \citet{Predehl2020} for modeling the eROSITA bubbles\
 based on X-ray emissions.

 The temperature profile (left middle panel in \Cref{fig__profile}) along the positive z-axis in\
 \Cref{fig__jetI5+ismSeed3-45deg} indicates that\
 most of the region between the forward shock and the turbulent zone\
 is around 0.3-0.4 keV, similar to that observed by \citet{Miller2016} and \citet{Kataoka2018}.

 Followed by the forward shock is a turbulent and hot plasma.\
 As shown in the temperature or density slices of the fiducial run,\
 the full extent of the turbulent plasma is approximately agree with\
 that of the \textit{Fermi} bubbles \citep{Su2010}.
 Also, the temperature of the plasma is as high as 2 keV, comparable to that inside the \textit{Fermi} bubbles\
 by observing X-ray absorption lines through the hot\
 gaseous halo along many different sight lines in the sky \citep{Miller_2013}.\
 We also note that the turbulence within the hot plasma is in pressure balance with the external medium,\
 suggesting the outer edge of the \textit{Fermi} bubbles\
 is a tangling contact discontinuity rather than a shock.

 The most interesting finding is that there is a innermost bubbles\
 (dashed box in left top panel of \Cref{fig__jetI5+ismSeed3-45deg})\
 extending out from the Galactic center on either side of the thin disk.\
 The innermost bubbles are cold (1-10 eV), dense ($10^{-4}$--$10^{-2}$ cm$^{-3}$),\
 and underpressured with respect to the turbulent plasma.\
 Moreover, the close-up view of profiles around $z=3.7$ kpc (right column in \Cref{fig__profile})
 demonstrates that there is a sharp pressure jump and dense shell at $z=3.61$ kpc,\
 indicating that the innermost bubbles are an expanding reverse shock.\
 We also find that the turbulent plasma is bracketed between the downstream of reverse shock\
 and of the outermost forward shock, thus heating the turbulent plasma up considerably.

 We stressed that either the outermost, turbulent plasma, or innermost bubbles\
 are symmetric about the Galactic plane despite the jet is tilted to the disk normal\
 at an angle $45^{\circ}$.

 To explore how the disk affect the formation of the Galactic bubbles,\
 we compare the fiducial run (first column from left in \Cref{fig__jetI5+ismSeed3-45deg})\
 with the case adopting the smooth disc\
 (second column; \Cref{fig__initial-density-profile} shows the density profile) in a stratified atmosphere.\
 The results show that the initial density distribution of the dense disc has an insignificant effect\
 on the overall dynamics of bubbles. However, the outermost bubbles arising from the smooth disc\
 in an uniform atmosphere (third column) is spherical-shape,\
 suggesting the stratification facilitates the outermost bubbles elongation significantly.\
 Also, the two rightmost columns reveal that the development of the innermost bubbles\
 is always associated with the disc, without the disc,\
 the outermost bubbles and the turbulent plasma will be oblique,\
 indicating the dense disc is crucial for\
 the symmetry of the Galactic bubbles and for the innermost bubbles formation.

 %originally driven by an oblique bipolar jets emanating from the GC 12.39 million years ago,\
 %and significantly stretched by the stratified atmosphere afterwards.

 % shock wave, followed by a contact discontinuity, moves to the right.

 % the innermost bubbles is an expanding reverse shock
 % the forward shock

  \begin{figure}
    \includegraphics[width=\columnwidth]{figures/fig__jetI5+ismSeed3-45deg.png}
    \caption{
             The slices of pressure (top), temperature (middle), and number density (bottom)\
             at the end of simulation $t=12.39$ Myr.\
             The slices pass through a bipolar jet source injecting along $z=-y$ direction\
             for a duration $t=0$--$1.2$ Myr.\
             Comparison between the clumpy (first column from left)\
             and the smooth disc (second) in a stratified atmosphere\
             shows that the initial density distribution of the dense disc has an insignificant effect\
             on the overall dynamics of bubbles. However, the outermost bubbles arising from the smooth disc\
             in an uniform atmosphere (third) is spherical-shape,\
             suggesting the stratification facilitates the outermost bubbles elongation significantly.\
             Also, the two rightmost columns reveal that the development of the innermost bubbles\
             (dashed box in top left panel)\
             is always associated with the disc, without the disc,\
             the outermost bubbles and the turbulent plasma will be oblique,
             indicating the dense disc is crucial for\
             the symmetry of the Galactic bubbles and for the innermost bubbles formation.
             }
    \label{fig__jetI5+ismSeed3-45deg}
  \end{figure}

  \begin{figure}
    \includegraphics[width=\columnwidth]{figures/fig__profile.png}
    \caption{
             Left: the profiles of pressure (top), temperature (middle),\
             and number density (bottom)\
             along the positive z-axis in \Cref{fig__jetI5+ismSeed3-45deg}.
             Right: the close-up view of profiles around $z=3.7$ kpc.\
             The sharp pressure jump and dense shell at $z=3.61$ kpc\
             indicate that the innermost bubbles\
             (dashed box in top left panel of \Cref{fig__jetI5+ismSeed3-45deg})\
             are an expanding reverse shock.\
     }
    \label{fig__profile}
  \end{figure}

  \subsection{X-ray}
  \label{X-ray}
  The X-ray emissivity is computed\
  for each computational volume cell
  using the MEKAL model \citep{Xray-1,Xray-2,Xray-3}\
  implemented in the utility XSPEC \citep{XSPEC}, assuming solar metallicity.\
  The X-ray intensity map are then generated by projecting the emissivities\
  along lines of sight\
  pointing away from the solar position at $(R_{\odot},0,0)=(8,0,0)$ kpc\
  with angular resolutions of 0.5 degrees, where $R_{\odot}$ is the Sun-GC distance.

  We point out that the projections throughout this paper is \lq perspective\rq,\
  which has the effect of making distant object appear smaller than the same object in the near distance,\
  in order to facilitate a reliable interpretation of simulated all-sky map,\
  and also that the observed X-ray emission is contributed by all the gas in the Milky Way halo,\
  which likely extends to a radius of $\sim$250 kpc \citep{halo-radius-1,halo-radius-2},\
  much bigger than our simulation box.

  We first compute the X-ray emissivity\
  from the simulated gas within a radius of 25 kpc away from the GC.
  Then, beyond 25 kpc the gas is assumed to be isothermal with $T=10^6$ K and\
  follows out to a radius of 250 kpc the observed density profile of \citep{temperature-MW}.

  \Cref{fig__xray_0.8keV_angle_000} shows\
  the comparison between simulated (top) and observed (bottom) all-sky map\
  in the range 0.6--1.0 keV at 12.39 Myr and the present time, respectively.\
  In the simulated map, the red arrow at the center of map represents the direction of the bipolar jet,\
  constantly ejecting at an angle of $45^{\circ}$ to the disc normal between 0--1.2 Myr.

  \Cref{fig__x-ray-profile-0.8keV-000} displays the simulated count rate profiles (red)\
  in the same energy band as \Cref{fig__xray_0.8keV_angle_000}\
  cut at various galactic latitudes (as labelled), compared with the observation (black).\
  \Cref{fig_xray-map} is quite revealing in several ways.\

  First, the broad agreement between simulated and observed X-ray maps hints that\
  the full vertical extent of the eROSITA bubbles can be properly formed by an inclined jet within a thin disk\
  of dense interstellar medium.\


  Second, as shown in \Cref{fig__jetI5+ismSeed3-45deg},\
  the half-width of the outermost bubbles is around 7 kpc,\
  corresponding to an half angular width $\sin^{-1}(7 \text{ kpc}/R_{\odot})\sim122^{\circ}$,\
  which is as wide as the eROSITA bubbles in the simulated X-ray map (top panel in \Cref{fig__xray_0.8keV_angle_000}).\
  We therefore suggest that the nature of eROSITA bubbles is indeed a forward shocks\
  that have been driven into the northern and the southern Galactic halo,
  as previously proposed by \citet{Predehl2020}.


% 2. North Polar Spur is not shown as it is a supernova remanent near us.
  Third, the innermost bubbles\
  shown in \Cref{fig__jetI5+ismSeed3-45deg},\
  even though with high column density, is invisible in the simulated X-ray map as\
  the temperature of innermost bubbles is around $1$--$10$ eV\
  (see the temperature profile in \Cref{fig__profile}).\
  Consequently, the X-ray emission within the innermost bubbles
  is severely suppressed by the cutoff $\exp\left[-h\nu/k_{B}T\right]$ in the thermal Bremsstrahlung emissivity.\
  This is the reason why the innermost bubbles is unseen in the observation.


  \begin{figure*}
    \subfigure[Simulated (top) and observed (bottom; \citealt{Predehl2020}) count rate\
               (photons s$^{-1}$ deg$^{-2}$) in the 0.6--1.0 keV range.\
               Throughout this paper we show sky maps\
               in Galactic coordinates centered on the Galactic center using a Hammer-Aitoff projection,\
               and observed from the solar system.\
               The red arrow at the center of the\
               top panel depicts the direction of the bipolar jet, constantly ejecting at an angle of $45^{\circ}$\
               to the disc normal in 1.2 Myr.]
     {
      \includegraphics[width=0.8\linewidth]{figures/fig__xraymap.png}
      \label{fig__xray_0.8keV_angle_000}
     }
    \subfigure[Comparison of simulated (red) and observed (black; \citealt{Predehl2020}) one-dimensional\
               count rate profiles in the same energy band as \Cref{fig__xray_0.8keV_angle_000},\
               cut at various galactic latitudes (as labelled).]
     {
      \includegraphics[width=0.8\linewidth]{figures/fig__xray-profile-0.8keV-000.png}
      \label{fig__x-ray-profile-0.8keV-000}
     }
    \caption{}
    \label{fig_xray-map}
  \end{figure*}



\subsection{Gamma-ray and microwave spectra: constraint on the CRe spectral index}

In this section, we give the constraint on the CRe spectral index by\
comparing the simulated gamma-ray and microwave spectra\
with the observed spectra of the Fermi bubbles \citep{Ackermann2014}\
and the microwave haze \citep{Dobler_2008}, respectively.\

The simulated gamma-ray and microwave spectra is based on the\
IC scattering and synchrotron from CRe, respectively, in which\
the CRe spectrum follows a power-law distribution ranging from\
$0.5$ MeV to $562.1$ GeV.

%The microwave haze at 23 GHz was first detected by the \textit{WMAP} \citep{Dobler_2008}\
%within the region $|b|<30^{\circ}$ above and below the Galactic plane,\
%and later confirmed by the Planck Collaboration IX \citep{PlanckCollaborationIX2013}.

The IC emissivity of the upscattered photons at the energy $\epsilon_{1}$ is computed for\
each computational cell in our simulations\
using the Klein-Nishina IC cross-section \citep{Jones1968,BLUMENTHAL1970}\
to handle the scattering between ultra-relativistic CRe and photons in the ISRF:

\begin{subequations}
  \begin{align}
  &\frac{dE}{dtd\epsilon_{1}dV} =\nonumber\\
               &\frac{3}{4}\sigma_{T}c\mathbb{C}\epsilon_{1}\int^{\epsilon_{\text{max}}}_{\epsilon_{\text{min}}}
               \frac{n(\epsilon)}{\epsilon}d\epsilon\int^{\gamma_{\text{max}}}_{\gamma_{\text{min}}\left(\epsilon\right)}
               \gamma^{-(p+2)}f(q, \Gamma)d\gamma,\\
  \nonumber\\
  &f(q, \Gamma) =\nonumber\\
               &2q\ln q+(1+2q)(1-q)+0.5(1-q)\frac{\left(\Gamma q\right)^2}{1+\Gamma q},\\
  &q=\frac{\epsilon_{1}/\gamma\
               m_{\text{e}}c^{2}}{\Gamma\left(1-\epsilon_{1}/\gamma m_{\text{e}}c^{2}\right)},\\
  &\Gamma=\frac{4\epsilon \gamma}{m_{\text{e}}c^2},\\
  &\gamma_{\text{min}}(\epsilon)=\
   0.5\left(\frac{\epsilon_{1}}{m_{\text{e}}c^2}+\sqrt{\left(\frac{\epsilon_{1}}{m_{\text{e}}c^2}\right)^2+\
   \frac{\epsilon_{1}}{\epsilon}}\right) \label{gamma-min},
  \end{align}
\label{gammaray-emissivity}
\end{subequations}

where $\sigma_{T}$ is the Thomson cross section, $c$ speed of light,\
$m_{\text{e}}$ electron mass,\
$n(\epsilon)$ the energy distribution of the photon number density in ISRF given by \citet{Porter2017},\
$\gamma$ the Lorentz factor of CRe,\
$\mathbb{C}$ and $p$ are the normalization constant and spectral index of CRe power-law spectrum.
$\gamma_{\text{min}}(\epsilon)$ is the minimum Lorentz factor of CRe\
that allows incident photons to scatter from energy $\epsilon$ to $\epsilon_{1}$.
$\gamma_{\text{max}}$ is the maximum CRe Lorentz factor in the spectrum.

We perform the double integration in \Cref{gammaray-emissivity} on each cell\
over the range of CRe Lorentz factor, and\
the range of incident photon energy between\
$\epsilon_{\text{min}}=1.13\times10^{-4}$ eV (cosmic microwave background) and\
$\epsilon_{\text{max}}=13.59$ eV (optical starlight)\
to obtain the simulated IC emissivities.


The synchrotron emissivity with an isotropic electron pitch angle distribution\
is given by \citet{BLUMENTHAL1970}:

\begin{subequations}
   \begin{align}
      &\frac{dE}{dtd\nu dV} =\nonumber\\
      &\frac{4\pi\mathbb{C}e^{3}B^{0.5(p+1)}}{m_{\text{e}}c^{2}}\
      \left(\frac{3e}{4\pi m_{\text{e}}c}\right)^{0.5(p-1)}\
      a(p)\nu^{-0.5(p-1)},\\
      &a(p)=\nonumber\\
           &\frac{2^{0.5(p-1)}\sqrt{3}\Gamma\left[\left(3p-1\right)/12\right]\
                                      \Gamma\left[\left(3p+9\right)/12\right]\
                                      \Gamma\left[\left(p+5\right)/4\right]}
      {8\sqrt{\pi}(p+1)\Gamma\left[\left(p+7\right)/4\right]},
   \end{align}
   \label{synchrotron-emissivity}
\end{subequations}

where $\Gamma$ is gamma function, $B$ is the magnetic field strength defined in\
\Cref{magnetic-field}.

For a given longitude and latitude range, the simulated spectra are\
computed by projecting emissivities\
as we project X-ray emissivities in Section \ref{X-ray},\
and then we average the spectra over all the sight lines within the region on sky.


\Cref{fig__gammaRaySynchtronSpectrum} shows the simulated microwave (left)\
and gamma-ray (right) spectra averaged over the different patches (shown in the legends) of the sky.\
The row from top to bottom shows the spectra with different CRe spectral index $2.2, 2.4$ and $2.6$.\
Several points are worth specific comment.\

First, we find that the simulated gamma-ray spectra are well fitted by\
the CRe spectral index 2.4 (the middle row),\
despite the simulated microwave spectra are marginally consistent with the observed.

Second, the simulated and observed gamma-ray spectra is\
nearly latitude independent, and characterized by a broad bump that roughly peaks around $\sim10$ GeV.\
However, the high-latitude spectrum tends to be slightly dimmer than low-latitude\
probably because the optical intensity in ISRF decays with increasing latitude.

Third, the gamma-ray spectra for all latitudes indicate a spectral cutoff around energies 400--500 GeV,\
remarkably consistent with the observed cutoff energy.\
This is expected since\
the upscattered high-energy photons ($\epsilon_{1}\sim450$ GeV) mainly arise from\
the scatterings between the relativistic CRe ($\gtrapprox 408$ GeV)\
and optical starlight ($\epsilon \sim 10$ eV),\
and therefore \Cref{gamma-min} can be reduced to $\epsilon_{1}\sim\gamma m_{\text{e}}c^2$\
in the Klein--Nishina limit\
$\left(\text{i.e. }\epsilon_{1}\epsilon \gg \left(m_{\text{e}}c^2\right)^2\right)$,\
implying most of the CRe energy is carried away by the upscattered photons.

Fourth, the good agreement between the simulated and observed gamma-ray/microwave spectra\
imply that the emission of the Fermi bubbles and\
the microwave haze can be produced by the same high-energy electrons\
via inverse Compton scatterings and synchrotron radiations\
in the presence of ISRF and magnetic fields, respectively,\
as previously suggested \citep{Su2010,Dobler2012}.

\begin{figure*}
  \includegraphics[width=\linewidth]{figures/fig__spectrum.png}
  \caption{
      Simulated microwave spectra (colored lines in left) averaged over $20^{\circ}<|b|<30^{\circ}$, $|l|<10^{\circ}$.\
      The data point represents the \textit{WMAP} data in the 23 GHz K band and\
      the shaded bow-tie area indicates the range\
      of synchrotron spectral indices allowed for the \textit{WMAP} haze \citep{Dobler_2008}.\
      Simulated gamma-ray spectra (colored lines in right column)\
      of the \textit{Fermi} bubbles calculated for a longitude range of\
      $|l|<10^{\circ}$ for different latitude bins.\
      The gray band represents the observational data of \citet{Ackermann2014}.\
      The row from top to bottom shows the microwave (left) and gamma-ray (right) spectra\
      with CRe spectral index $2.2, 2.4$ and $2.6$, respectively.\
      The CRe cutoff energy is $1.1\times10^{6}m_{\text{e}}c^{2}$ in all cases.
      %At larger $p$, the microwave spectrum was found to be steeper as $I_{\nu}\propto \nu^{-(p-1)/2}$.
  }
  \label{fig__gammaRaySynchtronSpectrum}
\end{figure*}

 Finally, we display the morphology of the \textit{Fermi} bubbles\
 that emit gamma-rays from the inverse Compton scattering of ISRF by CRe\
 with the best-fit CRe sptral index 2.4


 We close on a speculative note that predicts an existence of a third bubbles for
 future observations.

% assuming a uniform CR spectrum with energy cutoff of ~500 GeV\
% and spectral index of 2.4, one could also reproduce the gamma-ray spectrum quite well.\
% Note that the simulation time at this point is 12.39 Myr,\
% much longer than the CR electron cooling time, so their Emax should\
% have cooled to below 500 GeV by this time (we should think about how to estimate this).\
% Therefore, the CRe generating the gamma-ray emission would need to be re-accelerated\
% by some in-situ acceleration mechanisms such as shocks or turbulence.\
% The forward shock is pretty far away from the gamma-ray bubbles,\
% so it's more likely associated with turbulence.

%  Having successfully reproduced the morphology\
%  and spectrum of the \textit{Fermi} bubbles,\
%  we now ask whether CRe could also explain\
%  the Galactic microwave haze.

  %The enhanced emission at the lower latitude would likely be due to
  %similar to the bottom right panel in Fig. 5 of YR17, which is dominated by\
  %the optical starlight. That's also why in your plot above,\
  %the high-latitude spectrum tends to be dimmer because of the decay of the ISRF intensity in optical.


\begin{figure*}
  \includegraphics[width=\linewidth]{figures/fig__GammaRay_100e9_1e6_angle_000.png}
  \caption{The observed (left; \citealt{Selig2015}) and simulated (right) photon flux\
           in the energy bin $76.8-153.6$ GeV.\
           Note that the left panel is the\
           photon flux of the diffuse component reconstructed by the D$^3$PO\
           algorithm \citep{Selig2015} that analyzes\
           the photon data from the \textit{Fermi} Large Area Telescope \citep{Atwood2009}\
           and removes the contribution from point-like component.
           The red arrow at the center of the right\
           panel depicts the direction of the bipolar jet, constantly ejecting at an angle of $45^{\circ}$\
           to the disc normal in 1.2 Myr.
  }
  \label{fig__gammaRay-map}
\end{figure*}



  \begin{figure}
    \includegraphics[width=\columnwidth]{figures/fig__jetI5+ismSeed3-45deg-CR.png}
    \caption{The slice of CR... $p_{CR}$ is much less than $p_{\text{gas}}$...}
    \label{fig__jetI5+ismSeed3-45deg-CR}
  \end{figure}


\Cref{fig__gammaRay-map} shows the simulated gamma-ray photon flux compared with the observed one\
in the energy bin $76.8-153.6$ GeV. One can see that the symmetric of the \textit{Fermi} bubbles\
can also be realized by oblique jets as the eROSITA bubbles.\


%\subsubsection{Hadronic process}
%In the hadronic model, CRp undergo hadronic collisions with thermal gas protons\
%and produce $\gamma$-ray via pion decay. The volume emissivity of the emission can be written as
%\begin{equation}
%   \epsilon \propto U_{\text{CRp}}n_{p}\sigma_{p}\kappa_{pp}.
%\end{equation}
%
%\begin{equation}
%  \frac{dE}{dtd\epsilon_{1}dV} =\
%   cn_{H}pC\left(\frac{\epsilon_{1}}{m_{\text{p}}c^2}\right)^{-p}\
%   \int_{0}^{1}\sigma(\epsilon_{\text{p}}) F(x,\epsilon_{\text{p}}) x^{p}dx.
%\end{equation}
%
%\begin{equation}
%\sigma(\epsilon_{\text{p}})=34.3+1.88L+0.25L^{2}\left[1-\frac{E_{\text{threshold}}}{E_{\text{p}}}\right]^{4} \text{ mb}.
%\end{equation}
%
%\begin{equation}
%F(x,\epsilon_{\text{p}})=B\frac{d}{dx}\left[\ln(x)\left(\frac{1-x^{\beta}}{1+kx^{\beta}\left(1-x^{\beta}\right)}\right)^4\right],
%\end{equation}
%
%
%where
%$x=\epsilon_{1}/\epsilon_{\text{p}}$, $B=1.30+0.14L+0.011L^2$, $\beta=\left(1.79+0.11L+0.008L^{2}\right)^{-1}$,\
%$\left(0.801+0.049L+0.014L^{2}\right)^{-1}$, $L=\ln(\epsilon_{\text{p}}/1 \text{ TeV})$.



\section{Conclusions}
\label{Conclusions}
%The symmetry of the Galactic bubbles is worth exploring since observations\
%show that there is no preference for the orientation of AGN jets relative to\
%the rotation axis of a galactic plane.


In this work, we introduce a thin, dense disk composed of interstellar medium\
to divert the inclined jet at an angle $45^{\circ}$\
to the disk normal in the past energetic event of the central SMBH.


Also, We investigate the properties of the Galactic bubbles and the microwave\
haze using 3D special relativistic hydrodynamic simulations of CR injection from\
the SMBH assuming the leptonic model.


Our findings are as follows.

\begin{itemize}

\item The Galactic bubbles are nearly symmetric about the Galactic plane albeit the jet is at an angle $45^{\circ}$\
      with respect to the disc normal.\
      The broad agreement between simulated and observed mutiwavelength features\
      provides supporting evidence for the inclined jet scenario, and also\
      releases the caveat given by earlier simulation-based studies:\
      the jets shall be vertical to the disc normal.
\item The randomness of clumpy disk has a insignificant impact on the overall dynamic of the Galactic bubbles.\
      The ambient stratification facilitates the eROSITA bubbles elongation significantly.
      The development of the reverse shocks (the innermost bubbles) are always associated with the dense disc.
\item The edge of the eROSITA bubbles is a forward shock,\
      originally driven by an oblique bipolar jets emanating from the GC 12.39 million years ago,\
      and significantly stretched by the stratified atmosphere afterwards.
\item Followed by the forward shock is a tangling contact discontinuity,\
      which we interpret as the edge of the \textit{Fermi} bubbles.\
      The nature of the \textit{Fermi} bubbles is essentially a turbulent and high-temperature ($\sim2$ keV)\
      plasma in pressure balance with the external medium. Associated with the Galactic magnetic field\
      will lead to the stochastic acceleration of CRe, and probably balanced with the IC and synchrotron cooling.\
      We will investigate the competition between CRe acceleration and radiative cooling in a future work.
\item According to the microwave and leptonic gamma-ray spectra, the best-fitting\
      CRe power-law index is found to be -2.4.
\item The same leptotonic CRe can simultaneously\
      account for the \textit{Fermi} bubbles and haze emission suggested that\
      they are physically related around the GC,\
      and that the magnetic fields within the bubbles are close to the exponentially distributed\
      Galactic magnetic field.
\end{itemize}



%Over the past few years, there have been numerous efforts,\
%both observational and theoretical, to uncover the nature of the Galactic bubbles.
%
%
%Spektr-RG \citep{Sunyaev2021} and eROSITA \citep{Predehl2021}



We have found qualitative similarities between the results our simulations\
and the observed spectrum and morphological structure of the Galactic bubbles.


Since the jets encounter a larger column depth of clouds along their path,\
they interact more strongly with the dense ISM.\
The jets do not immediately clear a channel along the inclined axis of launch.\
Rather, the dense clouds deflect the jets, which then decelerate\
and launch a sub-relativistic outflow along the minor axis following the path of least resistance.

Assuming that a relativistic electron population of spectral
index x = 2.2 gives rise to the microwave haze via synchrotron
emission.


%We therefore suggest that the Fermi bubbles and the eROSITA structure are physically related,\
%as suggested by \citet{Predehl2020}.

%Our discovery confirms the previously suggested common origin of the two objects \citep{Sofue2000,Kataoka2018}

\section*{Data Availability}
The data underlying this article are available in the article and in its online supplementary material.


%%%%%%%%%%%%%%%%%%%% REFERENCES %%%%%%%%%%%%%%%%%%
\bibliographystyle{mnras}
\bibliography{paper} % if your bibtex file is called example.bib

%%%%%%%%%%%%%%%%% APPENDICES %%%%%%%%%%%%%%%%%%%%%

%\appendix
% https://heasarc.gsfc.nasa.gov/docs/objects/heapow/archive/normal_galaxies/fermibubbles_erosita.html

%Jet_SrcVel 0.6
%Jet_SrcDens 1e-26
%Jet_SrcTemp 2e10
%Jet_SrcPres => 1.6e-8

\end{document}
