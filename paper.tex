\documentclass[twocolumn]{aastex631}
\usepackage{mathtools}
\usepackage{natbib}
\bibliographystyle{abbrvnat}
\setcitestyle{authoryear,open={(},close={)}}
\usepackage{txfonts}
\usepackage{lipsum, babel}
\usepackage[T1]{fontenc}
\usepackage{ae,aecompl}
\usepackage{tikz}
\usepackage{graphicx}
\usepackage{url}
\usepackage{subfigure}
\usepackage{float}
\usepackage{amsmath}
\usepackage{amssymb}
\usepackage{cleveref}
\usepackage{physics}
\usepackage{empheq}
\usepackage{booktabs}
\usepackage{array}
\newcolumntype{R}[1]{>{\raggedleft\arraybackslash}p{#1}}
\newcolumntype{L}[1]{>{\raggedright\arraybackslash}p{#1}}
\usepackage{multirow}
\usepackage[flushleft]{threeparttable}
\usepackage{mathrsfs}
\usepackage{soul}
\pdfminorversion=5

\newcommand{\MyDiamond}[1][fill=black]{
\begin{tikzpicture}[x=1.2ex,y=1.2ex,line width=.1ex,line join=round, yshift=0.0ex] \draw  [#1]  (0,.5) -- (.5,1) -- (1,.5) -- (.5,0);
\end{tikzpicture}
}

\newcommand{\MyPlus}[1][fill=black]
{
\begin{tikzpicture}[x=1.5ex,y=1.5ex,line width=0.5ex]
\draw[#1] (0,.5) -- (1,.5);
\draw[#1] (.5,0) -- (.5,1);
\end{tikzpicture}
}

\newcommand{\MyCross}[1][fill=black]
{
\begin{tikzpicture}[x=1.ex,y=1.ex,line width=0.5ex]
\draw[#1] (0,0) -- (1,1);
\draw[#1] (0,1) -- (1,0);
\end{tikzpicture}
}

\newcommand{\MyTriangle}[1][fill=black]
{
\begin{tikzpicture}[x=1.ex,y=1.ex,line width=0.2ex]
\draw[#1] (0,1) -- (1,1);
\draw[#1] (1,1) -- (.5,0);
\draw[#1] (.5,0) -- (0,1);
\fill[#1] (0,1) -- (1,1) -- (.5,0) -- cycle;
\end{tikzpicture}
}

\newcommand{\MySolidLine}[1][fill=black]
{
\begin{tikzpicture}[x=2.ex,y=1.ex,line width=0.5ex]
\draw[#1] (0,.5) -- (2.5,.5);
\end{tikzpicture}
}

\newcommand{\MyDashedLine}[1][fill=black]
{
\begin{tikzpicture}[x=2.ex,y=1.ex,line width=0.5ex]
\draw [#1,dashed] (0,.5) -- (2.5,.5);
\end{tikzpicture}
}

\newcommand{\MyDashedDottedLine}[1][fill=black]
{
\begin{tikzpicture}[x=2.ex,y=1.ex,line width=0.5ex]
\draw [#1,dash dot] (0,.5) -- (2.5,.5);
\end{tikzpicture}
}

\newcommand{\MyDottedLine}[1][fill=black]
{
\begin{tikzpicture}[x=2.ex,y=1.ex,line width=0.5ex]
\draw [#1,dotted] (0,.5) -- (2.5,.5);
\end{tikzpicture}
}

\begin{document}
\title{Fermi and eROSITA bubbles}


\author[0000-0002-1868-0660]{Po-Hsun Tseng}
\affiliation{Institute of Astrophysics, National Taiwan University, Taipei 10617, Taiwan}

\author[0000-0003-3269-4660]{H.-Y. Karen Yang}
\affiliation{Institute of Astronomy and Department of Physics, National Tsing Hua University, Hsinchu 30013, Taiwan}

\author[0000-0002-1249-279X]{Hsi-Yu Schive}
\affiliation{Institute of Astrophysics, National Taiwan University, Taipei 10617, Taiwan}
\affiliation{Department of Physics, National Taiwan University, Taipei 10617, Taiwan}
\affiliation{Center for Theoretical Physics, National Taiwan University, Taipei 10617, Taiwan}
\affiliation{Physics Division, National Center for Theoretical Sciences, Taipei 10617, Taiwan}

\author[0000-0003-2654-8763]{Tzihong Chiueh}
\affiliation{Institute of Astrophysics, National Taiwan University, Taipei 10617, Taiwan}
\affiliation{Department of Physics, National Taiwan University, Taipei 10617, Taiwan}
\affiliation{Center for Theoretical Physics, National Taiwan University, Taipei 10617, Taiwan}


\correspondingauthor{Po-Hsun Tseng}
\email{zengbs@gmail.com}


\keywords{keywords}

\begin{abstract}
\end{abstract}

\section{Introduction}


\section{Methodology}
\subsection{Assumptions and Numerical Techniques}
\begin{enumerate}
  \item Numerical features:
    \begin{enumerate}
      \item GPU acceleration with \textsc{gamer-sr} code.
      \item Special relativistic hydrodynamics.
      \item We use a new algorithm \citep{tseng2021}, dedicated to the conversion between\
            primitive and conserved variables, to significantly reduce\
            numerical error caused by catastrophic cancellations that commonly occur\
            within the jet-ISM interaction zones.
    \end{enumerate}
  \item Assumptions on cosmic-ray:
    \begin{enumerate}
      \item We treat CRs as a second fluid and solve directly for the evolution of CR pressure $p_{\text{cr}}$\
            as a function of $\mathbf{r}$ and $t$.
      \item We did not model the CR energy spectrum.
      \item We neglected the cooling and heating processes of CRs, such as energy losses due to synchrotron\
            and inverse Compton emission, and reacceleration in shocks.
      \item We have assumed cosmic-ray is passive. (i.e. $p_{\text{cr}}\ll p_{\text{gas}}$)
      \item We have ignored the $\mathbf{B}$ field within\
            the simulation box as the field inside\
            the bubbles should be weak due to adiabatic expansion, and thus the magnetic fields has\
            little effect on the overall dynamics.\
            We also assumed the magnetic field is highly entangled in small-scale (how small?), resulting in\
            the negligible cosmic-ray diffusion.
    \end{enumerate}
  \item Assumptions on gravity:
    \begin{enumerate}
      \item We have assumed relativistic gravity is insignificant.
      \item The ISM disk and atmosphere are subjected to the fixed external potential\
            due to a disk bulge and dark matter halo.
      \item In addition to the gravitational interaction, we ignore other interactions between stars and gases.
      \item We also ignore the self-gravity of the ISM disk and of the atmosphere.
      \item We ignore the centrifugal force due to the rotation of Milky Way.
      \item We use the potential of isothermal slab to mimic the gravitational potential due to the stellar bulge.
      \item The interface between cold ISM disk and atmosphere is parallel to galactic plane and is pressure balanced.
    \end{enumerate}
\end{enumerate}
We simulate 3D special relativistic hydrodynamics with passive CR injections from the GC using\
the special hydrodynamics GPU code \textsc{gamer-sr} \citep{tseng2021}.

\textsc{gamer-sr} solves mass and energy-momentum conservation laws of a special relativistic ideal fluid\
with CR.

The CRs are advected with the thermal gas, but the gas cannot react to the CR pressure.\
In this approach, the CRs are treated as a single species without distinction between electrons and protons.\
We did not model the CR energy spectrum, and we neglected the cooling and heating processes of CRs,\
such as energy losses due to synchrotron and inverse Compton emission, and reacceleration in shocks.\

\begin{subequations}
  \label{conservative-form}
  \begin{align}
   &\partial_{t} D+\partial_{j} \left(DU^{j}/\gamma\right)=0,\label{D evolution}\\
   &\partial_{t} M^{i}+\partial_{j} \left(M^{i}U^{j}/\gamma+p_{\text{gas}}\delta^{ij}\right)=\
   -\rho\partial_{i}\Phi,\label{M evolution}\\
   &\partial_{t} \tilde{E}+\partial_j \left[\left(\tilde{E}+p_{\text{gas}}\right)U^{j}/\gamma\right]=0, \label{E evoltion}\\
   &\partial_{t} \left(\gamma e_{\text{cr}}\right) + \partial_{j} \left(e_{\text{cr}}U^{j}\right)=-p_{\text{cr}} \partial_{j} U^{j},\label{D evolution}
  \end{align}
\end{subequations}
where the five conserved quantities of gas $D$, $M^{i}$, and $\tilde{E}$ are the mass density,\
the momentum densities, and the reduced energy density, respectively.\
The reduced energy density is defined by subtracting the rest mass energy density of gas\
from the total energy density of gas.\
$\gamma$ and $U^{j}$ are the temporal and spatial component of four-velocity of gas.\
$\rho$ is the gas density in the local rest frame defined by $D/\gamma$.\
$p_{\text{gas}}$ is the gas pressure.\
$p_{\text{cr}}$ and $e_{\text{cr}}$ are the CR pressure and CR energy density measured in the local rest frame.\
$\Phi$ is the gravitation potential.\
$c$ is the speed of light, and $\delta^{ij}$ is the Kronecker delta notation.\
Throughout this paper, Latin indices run from 1 to 3, except when stated otherwise.

\subsection{The Galactic Model}


\begin{enumerate}
\item Fixed external gravitational potential:
  \begin{enumerate}
    \item Bulge potential:
      \begin{enumerate}
       \item Peak density: $\rho_{\text{bulge}}^{\text{peak}}=4\times 10^{-24}$ g/cm$^3$.
       \item Potential (isothermal slab):
         \begin{equation}
           \Phi_{\text{bulge}}=\
           2\sigma^2_{\text{bulge}}\
           \ln\cosh\left(z\sqrt{\frac{2\pi G\rho_{\text{bulge}}^{\text{peak}}}{\sigma^2_{\text{bulge}}}}\right),
         \end{equation}
          where $\displaystyle \sigma_{\text{bulge}}=\
                \sqrt{\frac{k_{B}T_{\text{bulge}}}{m}}=100$ km/s \citep{velocity-dispersion-MW}.
      \end{enumerate}

    \item Dark logarithmic halo potential:
         \begin{equation}
           \Phi_{\text{halo}}=v^2_{\text{halo}}\ln\left(z^2+d^2_{\text{h}}\right),
         \end{equation}
         where $v_{\text{halo}}=131.5$ km/s, $d_{\text{h}}=12$ kpc \citep{Yang2013}.

    \item Total potential: $\Phi_{\text{total}}=\Phi_{\text{bulge}}+\Phi_{\text{halo}}$.
  \end{enumerate}


\item Clumpy cold disk:
  \begin{enumerate}
    \item Dimension: $14\times14\times0.2$ kpc. i.e. Scale height $(z_{0})=100$ pc. \citep{paek-ism-density}
    \item Peak mass density: $\rho_{\text{disk}}^{\text{peak}}=10^{-23}$ g/cm$^3$. \citep{paek-ism-density}
    \item Temperature: $T_{\text{disk}}=10^{3}$ K. \citep{paek-ism-density}
    \item Average mass density:
          \begin{equation}
             \rho_{\text{disk}}=\rho_{\text{disk}}^{\text{peak}}
             \exp\left[-\frac{\Phi_{\text{total}}}{k_{B}T_{\text{disk}}/m}\right].
             \label{disk-density}
          \end{equation}

    \item Pressure: $\displaystyle p_{\text{disk}}=\rho_{\text{disk}}T_{\text{disk}}$.

    \item The clumpy cuboid, using the publicly\
          available pyFC code\footnote{\url{https://pypi.python.org/pypi/pyFC}},\
          are described by a log-normal distribution with mean 1.0 and variance 5.0.\
          Also, the power spectrum of the clumpy cuboid is characterized by $k_{\text{min}}$ and $\beta$,\
          where $k_{\text{min}}$ sets the maximum cloud size within the clumpy cuboid,\
          and $\beta$ is the slope of power spectrum in Fourier space.
          In this paper, we set $\beta=-5/3$ and $k_{\text{min}}=375$ to follow the Kolmogorov spectrum\
          and to limits the maximum size of an individual cloud to approximately 25 pc.

    \item The clumpy disk is then constructed by multiplying a fractal cuboid with \Cref{disk-density}.

    \item The clouds and voids within clumpy disk are pressure balanced.
  \end{enumerate}


\item Isothermal atmosphere:
  \begin{enumerate}
     \item The region outside the clumpy cold disk is atmosphere.
     \item $T_{\text{atmp}}=10^{6}$ K. \citep{temperature-MW}
     \item Density:
          \begin{equation}
             \rho_{\text{atmp}}=\rho_{\text{atmp}}^{\text{peak}}
             \exp\left[-\frac{\Phi_{\text{total}}}{k_{B}T_{\text{atmp}}/m}\right],
             \label{atmosphere-density}
          \end{equation}
          where $\rho_{\text{atmp}}^{\text{peak}}$ can be obtained by assuming that pressure\
          and gravitational potential are continuous at the interface ($z=\pm z_{0}$) between disk and atmosphere.
  \end{enumerate}



\end{enumerate}
\subsection{Jet injection}

\subsection{X-ray and Gamma-ray emission}
  \begin{enumerate}
    \item X-ray:
       \begin{enumerate}
         \item Thermal bremsstrahlung: The X-ray emissivity in an energy range 1.4–1.6 keV is calculated\
               using the MEKAL model (\citealt{Xray-1}; \citealt{Xray-2}; \citealt{Xray-3})\
               implemented in the utility XSPEC\citep{XSPEC}, assuming solar metallicity.
       \end{enumerate}
    \item Gamma-ray:
       \begin{enumerate}
         \item Hadronic process:
       \end{enumerate}
  \end{enumerate}


\section{conclusions}

\section*{Data Availability}
The data underlying this article are available in the article and in its online supplementary material.


%%%%%%%%%%%%%%%%%%%% REFERENCES %%%%%%%%%%%%%%%%%%
\bibliographystyle{mnras}
\bibliography{paper} % if your bibtex file is called example.bib

%%%%%%%%%%%%%%%%% APPENDICES %%%%%%%%%%%%%%%%%%%%%

\appendix

\end{document}
